\documentclass[a4paper, 12pt]{article}
\usepackage[T2A]{fontenc}
\usepackage[utf8]{inputenc}
\usepackage[english,russian]{babel}
\usepackage{titletoc}
\usepackage{graphicx}
\usepackage{array}
\usepackage{etoolbox}
\usepackage{subfig}
\usepackage{csvsimple}
\usepackage{float}
\usepackage{subcaption}
\usepackage{verbatim}
\usepackage{listings}
\usepackage{xcolor}
\usepackage[numbers]{natbib}
\usepackage{bibentry}
\usepackage{amsmath}
\usepackage{algorithm}
\usepackage{algpseudocode}
\usepackage[colorlinks,allcolors=black]{hyperref}
\definecolor{dkgreen}{rgb}{0,0.6,0}
\definecolor{gray}{rgb}{0.5,0.5,0.5}
\definecolor{mauve}{rgb}{0.58,0,0.82}
\definecolor{gray}{rgb}{0.4,0.4,0.4}
\definecolor{darkblue}{rgb}{0.0,0.0,0.6}
\definecolor{lightblue}{rgb}{0.0,0.0,0.9}
\definecolor{cyan}{rgb}{0.0,0.6,0.6}
\definecolor{darkred}{rgb}{0.6,0.0,0.0}

\include{includes/consts}
\usepackage[T2A]{fontenc}
\usepackage[utf8]{inputenc}
\usepackage[russian]{babel}

\usepackage[left=30mm, right=10mm, top=20mm, bottom=20mm]{geometry}

% Пакет для межстрочного интервала
\usepackage{setspace}
\onehalfspacing % Полуторный интервал

% Пакет для абзацного отступа
\usepackage{indentfirst}
\setlength{\parindent}{1.25cm} % Абзацный отступ 1.25 см

\addto{\captionsrussian}{\renewcommand{\contentsname}{Оглавление}}
\usepackage{listings}


\lstset{
	extendedchars=\true,
	inputencoding=utf8x,
	keepspaces=true,
	frame=tb,
	escapechar=`,
	xleftmargin=0.5cm,
	xrightmargin=0.5cm,
	columns=fixed,
	numbers=left,
	numbersep=4pt,
	showspaces=false,
	showstringspaces=false,
	breakatwhitespace=true,
	breaklines=true,
	basicstyle=\color{black}\small\sffamily,
	commentstyle=\color{gray}\itshape,
	stringstyle=\color{orange},
	numberstyle=\footnotesize\color{gray},
	keywordstyle=\color{blue}\bfseries,
	emphstyle={\color{blue}\bfseries},
	tabsize=2,
	texcl=true,
    mathescape=false
}

\lstset{
    language=bash,
    escapechar=`,
}

\lstdefinestyle{text}{
basicstyle=\small\ttfamily,
columns=flexible,
breaklines=true,
breakatwhitespace=true,
literate={\-}{{-\allowbreak}}1,
numbers=none,
postbreak=\mbox{\textcolor{red}{$\hookrightarrow$}\space},
mathescape=false
}

\lstdefinelanguage{cypher}
{
	morekeywords={
		MATCH, OPTIONAL, WHERE, NOT, AND, OR, XOR, RETURN, DISTINCT, ORDER, BY, ASC, ASCENDING, DESC, DESCENDING, UNWIND, AS, UNION, WITH, ALL, CREATE, DELETE, DETACH, REMOVE, SET, MERGE, SET, SKIP, LIMIT, IN, CASE, WHEN, THEN, ELSE, END,
		INDEX, DROP, UNIQUE, CONSTRAINT, EXPLAIN, PROFILE, START,
	}
}


\lstdefinelanguage{xml}
{
  morestring=[b]",
  morestring=[s]{>}{<},
  morecomment=[s]{<?}{?>},
  stringstyle=\color{black},
  identifierstyle=\color{darkblue},
  keywordstyle=\color{cyan},
  morekeywords={xmlns,version,type}% list your attributes here
}

\newcommand\YAMLcolonstyle{\color{red}\mdseries}
\newcommand\YAMLkeystyle{\color{black}\bfseries}
\newcommand\YAMLvaluestyle{\color{blue}\mdseries}

\makeatletter

% here is a macro expanding to the name of the language
% (handy if you decide to change it further down the road)
\newcommand\language@yaml{yaml}

\expandafter\expandafter\expandafter\lstdefinelanguage
\expandafter{\language@yaml}
{
  keywords={true,false,null,y,n},
  keywordstyle=\color{darkgray}\bfseries,
  basicstyle=\YAMLkeystyle,                                 % assuming a key comes first
  sensitive=false,
  comment=[l]{\#},
  morecomment=[s]{/*}{*/},
  moredelim=[l][\color{orange}]{\&},
  moredelim=[l][\color{magenta}]{*},
  moredelim=**[il][\YAMLcolonstyle{:}\YAMLvaluestyle]{:},   % switch to value style at :
  morestring=[b]',
  morestring=[b]",
  mathescape=false,
  literate =    {---}{{\ProcessThreeDashes}}3
                {>}{{\textcolor{red}\textgreater}}1     
                {|}{{\textcolor{red}\textbar}}1 
                {\ -\ }{{\mdseries\ -\ }}3,
}

% switch to key style at EOL
\lst@AddToHook{EveryLine}{\ifx\lst@language\language@yaml\YAMLkeystyle\fi}
\makeatother

\newcommand\ProcessThreeDashes{\llap{\color{cyan}\mdseries-{-}-}}

\lstdefinelanguage{docker}{
  keywords={FROM, RUN, COPY, ADD, ENTRYPOINT, CMD,  ENV, ARG, WORKDIR, EXPOSE, LABEL, USER, VOLUME, STOPSIGNAL, ONBUILD, MAINTAINER, HEALTHCHECK},
  identifierstyle=\color{black},
  sensitive=false,
  comment=[l]{\#},
  morestring=[b]',
  morestring=[b]"
}

\lstdefinelanguage{docker-compose}{
  keywords={image, environment, ports, container_name, ports, volumes, links},
  identifierstyle=\color{black},
  sensitive=false,
  comment=[l]{\#},
  morestring=[b]',
  morestring=[b]"
}
\lstdefinelanguage{docker-compose-2}{
  keywords={version, volumes, services},
  keywords=[2]{image, environment, ports, container_name, ports, links, build},
  keywordstyle=[2]\color{olive}\bfseries,
  identifierstyle=\color{black},
  sensitive=false,
  comment=[l]{\#},
  morestring=[b]',
  morestring=[b]"
}

\lstloadlanguages{SQL}

\newcolumntype{P}[1]{>{\centering\arraybackslash}p{#1}}
\def\textunderset#1#2{\leavevmode
  \vtop{\offinterlineskip\halign{%
    \hfil##\hfil\cr\strut#2\cr\noalign{\kern-.3ex}
    \hidewidth\strut#1\hidewidth\cr}}}
\def\signhrule{\raggedright\baselineskip30.0ex \vrule height 0.5pt width30mm depth0pt}

\dottedcontents{section}[2.5em]{\bfseries}{2em}{0.5pc}
\dottedcontents{subsection}[3em]{}{2em}{0.5pc}

\urlstyle{same}

%***natbib, bibentry***%
% Следующий код необходим для того, чтобы исправить конфликт между пакетами natbib+bibentry и стилем оформления ссылок согласно российскому ГОСТу cp1251gost705u
\ifx\undefined\selectlanguageifdefined
\def\selectlanguageifdefined#1{}\else\fi
\ifx\undefined\BibEmph
\def\BibEmph#1{\emph{#1}}\else\fi
\ifx\undefined\BibUrl
\def\BibUrl#1{\url{#1}}\else\fi
\ifx\undefined\BibAnnote
\def\BibAnnote#1{(#1)}\else\fi

\newcommand{\nonumsection}[1]{
\section*{#1}
\addcontentsline{toc}{section}{#1}
}
\begin{document}
\input{includes/titlepage}
\pagebreak
бла бла
\newpage
\tableofcontents
\newpage
\nonumsection{Введение}

Современные распределённые графовые базы данных сталкиваются с фундаментальной проблемой эффективного распределения вершин графа по узлам хранения (шардам). Оптимальное распределение становится критически важным для производительности систем, где основной операцией является поиск путей между вершинами, которые могут находиться в разных шардах. Неэффективное распределение приводит к значительным задержкам при выполнении запросов и избыточным сетевым коммуникациям между узлами.

Актуальность данной работы обусловлена стремительным ростом объёмов графовых данных в таких областях, как социальные сети, рекомендательные системы, биоинформатика и интернет вещей. Традиционные подходы к распределению данных демонстрируют ограниченную эффективность при работе с графами, требующими учёта структурных особенностей и связности вершин.

В рамках исследования проводится сравнительный анализ двух принципиально различных подходов к распределению графов: потоковых методов (online partitioning), работающих в реальном времени по мере поступления данных, и методов оптимизации распределения (offline partitioning), требующих полного знания структуры графа. Особое внимание уделяется алгоритмам библиотеки METIS, представляющей собой промышленный стандарт для задач разбиения графов, и современным потоковым алгоритмам, таким как Fennel и Streaming Graph Partitioning.

Целью работы является исследование и сравнение эффективности различных методов распределения вершин графа по гомогенным хранилищам, а также разработка предложений по комбинированию подходов для достижения оптимального баланса между качеством разбиения и вычислительной эффективностью в условиях реальной эксплуатации распределённых графовых баз данных.
\newpage
\section{Теоретическая часть}
\subsection{Постановка задачи распределения графа}

Формальная постановка задачи, рассматриваемой в работе, может быть сформулирована следующим образом. Дан граф $G = (V, E)$, где $|V| = n$ - количество вершин, $|E| = m$ - количество рёбер. Распределение графа представляет собой разбиение $P = \{ S_1, S_2, \dots, S_k \}$, где $S_i$ - набор вершин (шард) такой, что $S_i \cap S_j = \emptyset$ для $i \neq j$ и $\bigcup_{i=1}^k S_i = V$.



Требуется найти такое распределение $P^* = \{ S_1^*, S_2^*, \dots, S_k^* \}$, которое:
\begin{enumerate}
    \item Минимизирует общую мощность разрезов:
    \begin{equation}
        |\partial e(P)| = \left| \bigcup_{i=1}^k e(S_i^*, V \setminus S_i^*) \right| \rightarrow \min
    \end{equation}
    или относительную величину:
    \begin{equation}
        \lambda = \frac{|\partial e(P)|}{m} \times 100\% \rightarrow \min
    \end{equation}
    
    \item Минимизирует нормализованную максимальную нагрузку (максимизирует балансировку):
    \begin{equation}
        \rho = \frac{\max_{i=1..k}(|S_i^*|)}{\frac{n}{k}} \rightarrow \min
    \end{equation}
\end{enumerate}

Эта задача относится к классу NP-сложных задач, что обуславливает необходимость использования эвристических подходов, среди которых алгоритм Кернигана-Лина занимает важное место.

\subsection{Методы распределения вершин}

В рамках исследования рассматривались две принципиально различные категории методов решения задачи распределения:

\begin{enumerate}
    \item \textbf{Методы потокового распределения (online partitioning)}: Методы, распределяющие вершину исходя из данных только об уже распределённых вершинах. Эти методы могут быть использованы при изначальном наполнении базы данных.
    
    \item \textbf{Методы оптимизации распределения (offline partitioning)}: Методы распределения графа целиком, которые могут применяться для перераспределения графа по шардам, оптимизируя распределение, полученное методами первой категории.
\end{enumerate}

Алгоритм Кернигана-Лина относится ко второй категории и является методом \emph{уточнения} (refinement) разбиения. Как отмечается в исследовании, методы второй категории более точны и дают лучшие результаты за счёт знания всей структуры графа, но требуют больше вычислительных ресурсов.

В контексте библиотеки METIS (MEtis Tournament Inspired Strategy), которая рассматривается в работе как промышленный стандарт для задач разбиения графов, алгоритм Кернигана-Лина используется на этапе уточнения разбиения в рамках многоуровневой парадигмы.

\subsection{Многоуровневая парадигма и роль алгоритма KL}

Многоуровневая парадигма, лежащая в основе METIS, состоит из трёх этапов:

\begin{enumerate}
    \item \textbf{Свёртка (Coarsening)}: Последовательное уменьшение графа путём схлопывания вершин.
    
    \item \textbf{Распределение (Partitioning)}: Нахождение начального разбиения для сильно свёрнутого графа.
    
    \item \textbf{Развёртка и уточнение (Uncoarsening and Refinement)}: Последовательное восстановление исходного графа с одновременным улучшением разбиения.
\end{enumerate}

На третьем этапе алгоритм Кернигана-Лина играет ключевую роль. 

\subsection{Алгоритм Кернигана-Лина}
Алгоритм Кернигана-Лина (Kernighan-Lin, KL) --- это эвристический алгоритм для задачи разбиения графов на две равные по размеру части с минимальным весом разреза.

\textbf{Постановка задачи}

Пусть дан неориентированный взвешенный граф $G=(V,E)$ с весовой функцией $w: E \rightarrow {R}^+$. Требуется разбить множество вершин $V$ на два непересекающихся подмножества $A$ и $B$ таких, что:
\begin{itemize}
    \item $|A| = |B| = n/2$ (предполагаем, что $n = |V|$ чётно)
    \item Вес разреза минимален: $\min \sum_{a \in A, b \in B, (a,b) \in E} w(a,b)$
\end{itemize}

\textbf{Основные определения}

\begin{itemize}
    \item $P[v]$ --- разбиение, содержащее вершину $v$
    \item Для вершины $v$ определим \textbf{внешнюю стоимость}:
    \[
    E_v = \sum_{\substack{(v,u) \in E \\ P[v] \neq P[u]}} w(v,u)
    \]
    \item \textbf{Внутреннюю стоимость}:
    \[
    I_v = \sum_{\substack{(v,u) \in E \\ P[v] = P[u]}} w(v,u)
    \]
    \item \textbf{Функция улучшения} для вершины $v$ (разность между внешней и внутренней стоимостью):
    \begin{equation}
    g_v = E_v - I_v = \sum_{\substack{(v,u) \in E \\ P[v] \neq P[u]}} w(v,u) - \sum_{\substack{(v,u) \in E \\ P[v] = P[u]}} w(v,u)
    \end{equation}
\end{itemize}

Положительное значение $g_v$ означает, что перемещение вершины $v$ в противоположное разбиение уменьшит вес разреза.

\begin{algorithm}[H]
\caption{Алгоритм Кернигана-Лина}
\begin{algorithmic}[1]
\Require Граф $G=(V,E)$ с весами рёбер, начальное разбиение $(A,B)$
\Ensure Улучшенное разбиение $(A,B)$
\State Инициализация: $\text{улучшение} \gets \text{true}$
\While{$\text{улучшение} = \text{true}$}
    \State $A' \gets A$, $B' \gets B$ \Comment{Рабочие копии}
    \State Рассчитать $g_v$ для всех $v \in V$
    \For{$k \gets 1$ to $n/2$}
        \State Найти пару $(a_k, b_k)$ с максимальным выигрышем:
        \[
        g_k = \max_{\substack{a \in A' \\ b \in B'}} \left[ g_a + g_b - 2w(a,b) \right]
        \]
        где $w(a,b) = 0$, если $(a,b) \notin E$
        \State Пометить $a_k$ и $b_k$ как "заблокированные" (больше не участвуют в выборе на этой итерации)
        \State Обновить значения $g_v$ для соседей $a_k$ и $b_k$:
        \For{каждого соседа $x$ вершины $a_k$}
            \If{$x$ не заблокирован}
                \State $g_x \gets g_x + 2w(a_k,x)$ если $P[x] = P[a_k]$, иначе $g_x \gets g_x - 2w(a_k,x)$
            \EndIf
        \EndFor
        \For{каждого соседа $y$ вершины $b_k$}
            \If{$y$ не заблокирован}
                \State $g_y \gets g_y + 2w(b_k,y)$ если $P[y] = P[b_k]$, иначе $g_y \gets g_y - 2w(b_k,y)$
            \EndIf
        \EndFor
    \EndFor
    \State Найти $m$, максимизирующий частичную сумму:
    \[
    G_m = \max_{1 \leq m \leq n/2} \sum_{i=1}^m g_i
    \]
    \If{$G_m > 0$}
        \State Произвести обмены первых $m$ пар: $(a_1,b_1), \dots, (a_m,b_m)$
        \State $A \gets (A \setminus \{a_1,\dots,a_m\}) \cup \{b_1,\dots,b_m\}$
        \State $B \gets (B \setminus \{b_1,\dots,b_m\}) \cup \{a_1,\dots,a_m\}$
    \Else
        \State $\text{улучшение} \gets \text{false}$ \Comment{Улучшение не найдено}
    \EndIf
\EndWhile
\State \Return $(A,B)$
\end{algorithmic}
\end{algorithm}

\paragraph{Выигрыш от обмена пары вершин}
При обмене вершин $a \in A$ и $b \in B$:
\[
\Delta g(a,b) = g_a + g_b - 2w(a,b)
\]
где:
\begin{itemize}
    \item $g_a + g_b$ --- выигрыш от индивидуального перемещения вершин
    \item $-2w(a,b)$ --- коррекция, так как ребро между $a$ и $b$ (если существует) изменит свой статус: если оно было внутренним, станет внешним, и наоборот
\end{itemize}

\paragraph{Обновление $g_v$ для соседей}
После выбора пары $(a_k, b_k)$ и их блокировки, для соседей этих вершин значения $g$ обновляются:
\begin{itemize}
    \item Для соседа $x$ вершины $a_k$:
    \[
    g_x \gets 
    \begin{cases}
    g_x + 2w(a_k,x), & \text{если } x \text{ находится в том же разбиении, что и } a_k \\
    g_x - 2w(a_k,x), & \text{если } x \text{ находится в противоположном разбиении}
    \end{cases}
    \]
    \item Аналогично для соседей $b_k$
\end{itemize}

\textbf{Сложность алгоритма}

При наивной реализации сложность составляет $O(n^3)$:
\begin{itemize}
    \item Внешний цикл: $O(\text{итераций})$ (Экспериментально было установлено, что хорошая связность достигается уже после 5-10 итераций)
    \item Внутренний цикл по $k$: $O(n/2)$
    \item Поиск оптимальной пары на каждом шаге: $O(n^2)$
    \item Обновление значений $g$: $O(\text{степень вершины})$
\end{itemize}

С оптимизациями можно достичь сложности $O(n^2 \log n)$.

\textbf{Пример работы}

Рассмотрим простой граф из 6 вершин. Начальное разбиение: $A = \{1,2,3\}$, $B = \{4,5,6\}$.

\begin{enumerate}
    \item Рассчитываем $g_v$ для всех вершин
    \item Находим пару с максимальным $\Delta g(a,b)$
    \item Блокируем выбранные вершины, обновляем $g$ для их соседей
    \item Повторяем шаги 2-3, пока все вершины не будут заблокированы
    \item Находим $m$, максимизирующее частичную сумму выигрышей
    \item Если максимальная сумма положительна, производим обмены
\end{enumerate}


\subsubsection{Граничная модификация (BKL)}

Особый интерес представляет граничная модификация алгоритма (Boundary KL - BKL), которая применяет KL только к "граничным" вершинам - тем, которые смежны с вершинами из другого шарда. Как отмечается в работе: "Так как итерации прерываются, большинство вычислений тратятся впустую, особенно в KL(1), поэтому предлагается применять KL только к граничным вершинам. Это позволяет сильно уменьшить вычислительные затраты."

Результаты эксперимента (Таблица 4) показывают, что BKL(1) демонстрирует лучшее соотношение время/качество:
\begin{itemize}
    \item BCSSTK31: $\lambda = 8.08\%$, время выполнения = 0.76
    \item BCSSTK32: $\lambda = 7.31\%$, время выполнения = 0.96
\end{itemize}

\section{Практическая часть}

В рамках данной курсовой работы была реализована вычислительная модель для расчёта метрики улучшения распределения $g_v$ для граничных вершин графа согласно алгоритму Кернигана-Лина. Реализация включает в себя три основных компонента:
\begin{enumerate}
    \item Классы для представления графовых структур (вершины, рёбра, ключи)
    \item Класс хранилища (Storage) для управления вершинами и их связями
    \item Класс оптимизатора (StorageOptimizer) для расчёта метрики $g_v$
\end{enumerate}

Основной задачей практической части являлось создание инфраструктуры для вычисления функции улучшения распределения, определенной в алгоритме Кернигана-Лина:
\[
g_v = \sum_{\substack{(v,u) \in E \\ P[v] \neq P[u]}} w(v,u) - \sum_{\substack{(v,u) \in E \\ P[v] = P[u]}} w(v,u)
\]

\subsection{Структура проекта и основные зависимости}

Языком разработки был выбран С++ в силу его низкоуровневости и скорости, а также с намереньем в дальнейшим внедрить эти наработки в графовую БД на С++ научного руководителя.

Проект организован в виде набора заголовочных файлов (header files), что соответствует современным подходам разработки на C++. Основные файлы проекта:

\begin{itemize}
    \item \texttt{graph.hpp} -- содержит базовые классы для представления графовых структур
    \item \texttt{storage.hpp} -- реализует класс хранилища для управления вершинами
    \item \texttt{optimizer.hpp} -- содержит реализацию оптимизатора с вычислением метрики $g_v$
    \item \texttt{main.cpp} -- демонстрационный файл с тестовым сценарием
\end{itemize}

Ниже представлен релевантный для решения задачи курсовой работы код. Полный код представлен в приложении А.

\subsection{Классы для представления графовых структур (graph.hpp)}

\subsubsection{Класс NodeKey}

Класс \texttt{NodeKey} представляет собой обёртку для ключа вершины графа. Он обеспечивает типобезопасность и возможность использования различных типов данных в качестве ключей (целые числа, строки и т.д.).

\begin{lstlisting}[language=C++, caption=Класс NodeKey в graph.hpp, label=lst:nodekey]
template <typename KeyType> 
class NodeKey {
public:
    KeyType key_value;
    NodeKey(): key_value(KeyType()) {};
    NodeKey(const KeyType& key): key_value(key) {};

    bool operator<(const KeyType& other){
        return key_value < other;
    };

    bool operator>(const KeyType& other){
        return key_value > other;
    };

    bool operator==(const KeyType& other){
        return key_value == other;
    };

    NodeKey<KeyType>& operator=(const NodeKey<KeyType>& other) {
        key_value = other.key_value;
        return *this;
    };
};
\end{lstlisting}

Класс \texttt{NodeKey} является шаблонным, что позволяет использовать различные типы данных в качестве ключей вершин. Это важно для обеспечения гибкости при работе с различными типами графовых данных. Известно, что в конечной реализации используются строковые ключи, но была добавлена гибкость для потенциального использования пользовательских гибридных ключей.

\subsubsection{Класс Edge}

Класс \texttt{Edge} представляет ребро графа с весом и дополнительными параметрами.

\begin{lstlisting}[language=C++, caption=Класс Edge в graph.hpp, label=lst:edge]
class Edge {
public:
    int weight;
    std::map<std::string, Parameter> parameters;

    Edge& operator=(const Edge& other){
        weight = other.weight;
        parameters = other.parameters;
        return *this;
    };
};
\end{lstlisting}

\subsubsection{Класс Node}

Класс \texttt{Node} представляет вершину графа и содержит всю информацию о её связях с другими вершинами.

\begin{lstlisting}[language=C++, caption=Класс Node в graph.hpp (часть 1), label=lst:node1]
template <typename KeyType>
class Node {
public:
    typedef std::pair<NodeKey<KeyType>,Edge> Neighbour;

    Node(): key(NodeKey<KeyType>()) {};
    Node(KeyType _key): key(NodeKey<KeyType>(_key)) {};
    Node(NodeKey<KeyType> _key): key(_key) {};
    Node(NodeKey<KeyType> _key, std::vector<Neighbour> _this_storage_neighbours): 
        key(_key), this_storage_neighbours(_this_storage_neighbours) {};
    Node(NodeKey<KeyType> _key, std::vector<Neighbour> _this_storage_neighbours, 
         std::map<int,std::vector<Neighbour>> _other_storages_neighbours)
        : key(_key), 
          this_storage_neighbours(_this_storage_neighbours), 
          other_storages_neighbours(_other_storages_neighbours) {};
    
    // Конструктор копирования
    Node(const Node& other) 
        : key(other.key),
          parameters(other.parameters),
          this_storage_neighbours(other.this_storage_neighbours),
          other_storages_neighbours(other.other_storages_neighbours) {}
\end{lstlisting}

Класс имеет несколько конструкторов для удобного создания вершин с различными конфигурациями связей. Внутренние связи хранятся в \texttt{this\_storage\_neighbours}, а внешние связи (в другие хранилища) -- в \texttt{other\_storages\_neighbours}.

\begin{lstlisting}[language=C++, caption=Класс Node в graph.hpp (часть 2), label=lst:node2]
    NodeKey<KeyType> key;
    std::map<std::string, Parameter> parameters;

    std::vector<Neighbour> this_storage_neighbours;
    std::map<int,std::vector<Neighbour>> other_storages_neighbours;

    void set_this_storage_neighbours(std::vector<Neighbour> neighbours) const {
        this_storage_neighbours = neighbours;
    };
    
    void set_other_storages_neighbours(std::map<int,std::vector<Neighbour>> neighbours) const {
        other_storages_neighbours = neighbours;
    };
\end{lstlisting}

Структура данных организована так, чтобы эффективно разделять внутренние и внешние связи. Внешние связи организованы в виде отображения идентификатора хранилища на список соседей в этом хранилище.

\begin{lstlisting}[language=C++, caption=Методы класса Node для расчёта весов рёбер, label=lst:node3]
    // Получить сумму весов всех внутренних рёбер
    int get_internal_edges_weight_sum() const {
        int total_weight = 0;
        
        typename std::vector<Neighbour>::const_iterator it;
        for (it = this_storage_neighbours.begin(); 
             it != this_storage_neighbours.end(); ++it) {
            total_weight += it->second.weight;
        }
        
        return total_weight;
    }

    // Получить сумму весов внешних рёбер в конкретное хранилище
    int get_external_edges_weight_sum_to_storage(int target_storage_id) const {
        int total_weight = 0;
        
        typename std::map<int, std::vector<Neighbour>>::const_iterator map_it;
        map_it = other_storages_neighbours.find(target_storage_id);
        
        if (map_it != other_storages_neighbours.end()) {
            const std::vector<Neighbour>& edges = map_it->second;
            
            typename std::vector<Neighbour>::const_iterator edge_it;
            for (edge_it = edges.begin(); edge_it != edges.end(); ++edge_it) {
                total_weight += edge_it->second.weight;
            }
        }
        
        return total_weight;
    }
\end{lstlisting}

Методы \texttt{get\_internal\_edges\_weight\_sum()} и \texttt{get\_external\_edges\_weight\_sum\_to\_storage()} являются ключевыми для реализации алгоритма Кернигана-Лина, так как они непосредственно вычисляют суммы весов рёбер, необходимые для расчёта метрики $g_v$.

\subsection{Класс хранилища (storage.hpp)}

Класс \texttt{Storage} представляет собой хранилище вершин графа и реализует логику управления внутренними и внешними связями.

\subsubsection{Структура класса Storage}

\begin{lstlisting}[language=C++, caption=Базовая структура класса Storage, label=lst:storage1]
template <typename KeyType>
class Storage {
private:
    typedef Node<KeyType> StorageNode;
    int storage_id;
    std::unordered_map<KeyType, StorageNode> nodes;

public:
    Storage(int id) : storage_id(id) {}
    Storage(int id, std::unordered_map<KeyType, StorageNode> _nodes) 
        : storage_id(id), nodes(_nodes) {}
    
    int get_id() const {
        return storage_id;
    }
\end{lstlisting}

Хранилище идентифицируется уникальным \texttt{storage\_id} и содержит вершины в виде хэш-таблицы для обеспечения быстрого доступа по ключу.

\subsubsection{Добавление вершин с автоматическим созданием связей}

\begin{lstlisting}[language=C++, caption=Метод add\_node класса Storage, label=lst:storage2]
    bool add_node(const StorageNode& node) {
        KeyType key = node.key.key_value;
        typename std::unordered_map<KeyType, StorageNode>::iterator it = 
            nodes.find(key);
        if (it != nodes.end()) {
            return false;
        }
        nodes[key] = node;

        // Проходим по всем указанным соседям
        for (size_t i = 0; i < node.this_storage_neighbours.size(); ++i) {
            KeyType neighbor_key = node.this_storage_neighbours[i].first.key_value;
            Edge edge = node.this_storage_neighbours[i].second;
            
            // Пропускаем петли (ребра к самому себе)
            if (neighbor_key == key) {
                continue;
            }
            
            typename StorageNode::Neighbour neighbor_to(NodeKey<KeyType>(key), edge);
            if (has_node(neighbor_key)) {
                StorageNode* neighbor_node = get_node(neighbor_key);
                neighbor_node->this_storage_neighbours.push_back(neighbor_to);             
            }
        }
        return true;
    }
\end{lstlisting}

Метод \texttt{add\_node} не только добавляет вершину в хранилище, но и автоматически создает обратные связи с её соседями, что обеспечивает целостность графовой структуры. Это требуется для потокового распределения.

\subsubsection{Методы для работы с граничными вершинами}

\begin{lstlisting}[language=C++, caption=Методы для получения граничных вершин, label=lst:storage3]
    // Получить подграф узлов, имеющих соседей в указанном хранилище (копии)
    std::vector<StorageNode> get_nodes_with_neighbors_in_storage_copy(
        int target_storage_id) const {
        std::vector<StorageNode> result;
        
        typename std::unordered_map<KeyType, StorageNode>::const_iterator it;
        for (it = nodes.begin(); it != nodes.end(); ++it) {
            const StorageNode& node = it->second;
            
            typename std::map<int, std::vector<typename StorageNode::Neighbour>>::const_iterator map_it;
            map_it = node.other_storages_neighbours.find(target_storage_id);
            
            if (map_it != node.other_storages_neighbours.end()) {
                if (!map_it->second.empty()) {
                    result.push_back(node);
                }
            }
        }
        
        return result;
    }

    // Получить подграф узлов, имеющих соседей в указанном хранилище (копии)
    std::unordered_map<KeyType, StorageNode> 
    get_nodes_with_neighbors_in_storage_map_copy(int target_storage_id) const {
        std::unordered_map<KeyType, StorageNode> result;
        
        typename std::unordered_map<KeyType, StorageNode>::const_iterator it;
        for (it = nodes.begin(); it != nodes.end(); ++it) {
            const StorageNode& node = it->second;
            
            typename std::map<int, std::vector<typename StorageNode::Neighbour>>::const_iterator map_it;
            map_it = node.other_storages_neighbours.find(target_storage_id);
            
            if (map_it != node.other_storages_neighbours.end()) {
                if (!map_it->second.empty()) {
                    result.insert(*it);
                }
            }
        }
        
        return result;
    }
\end{lstlisting}

Эти методы реализуют важную оптимизацию алгоритма Кернигана-Лина -- работу только с граничными вершинами (Boundary KL Refinement). Это значительно снижает вычислительную сложность, так как исключает из рассмотрения вершины, не имеющие внешних связей.

\subsubsection{Методы для расчёта суммарных весов рёбер}

\begin{lstlisting}[language=C++, caption=Методы для расчёта весов рёбер в хранилище, label=lst:storage4]
    // Подсчет суммы весов всех внутренних ребер в хранилище
    int get_internal_edges_weight_sum() const {
        int total_weight = 0;
        int duplicate_weight = 0;
        
        typename std::unordered_map<KeyType, StorageNode>::const_iterator node_it;
        for (node_it = nodes.begin(); node_it != nodes.end(); ++node_it) {
            const StorageNode& node = node_it->second;
            
            typename std::vector<typename StorageNode::Neighbour>::const_iterator neighbour_it;
            for (neighbour_it = node.this_storage_neighbours.begin();
                neighbour_it != node.this_storage_neighbours.end();
                ++neighbour_it) {
                
                // пропускаем петли
                if (node.key.key_value == neighbour_it->first.key_value) {
                    continue;
                }

                total_weight += neighbour_it->second.weight;
                if (has_node(neighbour_it->first.key_value)) {
                    duplicate_weight += neighbour_it->second.weight;
                }
            }
        }
        
        return total_weight - duplicate_weight / 2;
    }

    // Подсчет суммы весов ребер из текущего хранилища в целевое хранилище
    int get_external_edges_weight_sum_to_storage(int target_storage_id) const {
        int total_weight = 0;
        
        typename std::unordered_map<KeyType, StorageNode>::const_iterator node_it;
        for (node_it = nodes.begin(); node_it != nodes.end(); ++node_it) {
            const StorageNode& node = node_it->second;
            
            typename std::map<int, std::vector<typename StorageNode::Neighbour>>::const_iterator map_it;
            map_it = node.other_storages_neighbours.find(target_storage_id);
            
            if (map_it != node.other_storages_neighbours.end()) {
                const std::vector<typename StorageNode::Neighbour>& edges = map_it->second;
                
                typename std::vector<typename StorageNode::Neighbour>::const_iterator edge_it;
                for (edge_it = edges.begin(); edge_it != edges.end(); ++edge_it) {
                    total_weight += edge_it->second.weight;
                }
            }
        }
        
        return total_weight;
    }
\end{lstlisting}


\subsection{Класс оптимизатора (optimizer.hpp)}

Класс \texttt{StorageOptimizer} является центральным компонентом реализации алгоритма Кернигана-Лина и отвечает за расчёт метрики улучшения $g_v$.

\subsubsection{Структура класса оптимизатора}

\begin{lstlisting}[language=C++, caption=Класс StorageOptimizer, label=lst:optimizer1]
template <typename KeyType>
class StorageOptimizer {
private:
    const Storage<KeyType>& storage1;
    const Storage<KeyType>& storage2;
    
public:
    StorageOptimizer(const Storage<KeyType>& s1, const Storage<KeyType>& s2)
        : storage1(s1), storage2(s2) {}
\end{lstlisting}

\subsubsection{Основной метод расчёта метрик}

\begin{lstlisting}[language=C++, caption=Методы расчёта g\_v, label=lst:optimizer2]
    int calculate_gv(const Node<KeyType>& node, int other_storage_id) const {
        int internal_edges_weight = node.get_internal_edges_weight_sum();
        int external_edges_weight = node.get_external_edges_weight_sum_to_storage(
            other_storage_id);
        
        return internal_edges_weight - external_edges_weight;
    }
    void calculate_gvs() const {
        // Получаем граничные вершины для обоих хранилищ
        std::unordered_map<KeyType, Node<KeyType>> boundary_nodes1 = 
            storage1.get_nodes_with_neighbors_in_storage_map_copy(storage2.get_id());
        std::unordered_map<KeyType, Node<KeyType>> boundary_nodes2 = 
            storage2.get_nodes_with_neighbors_in_storage_map_copy(storage1.get_id());

        typename std::unordered_map<KeyType, Node<KeyType>>::const_iterator it;
        for (it = boundary_nodes1.begin(); it != boundary_nodes1.end(); ++it) {
            const Node<KeyType>& node = it->second;
            
            std::cout << "Metric for node " << node.key.key_value << " is " 
                      << calculate_gv(node, storage2.get_id()) << std::endl;
        }

        for (it = boundary_nodes2.begin(); it != boundary_nodes2.end(); ++it) {
            const Node<KeyType>& node = it->second;
            
            std::cout << "Metric for node " << node.key.key_value << " is " 
                      << calculate_gv(node, storage1.get_id()) << std::endl;
        }
    }
\end{lstlisting}

Метод \texttt{calculate\_gvs()} демонстрирует практическую реализацию оптимизации Boundary KL (алгоритм работы только с граничными вершинами). Он получает граничные вершины из обоих хранилищ и вычисляет для каждой из них метрику улучшения $g_v$, а метод \texttt{calculate\_gv} занимается непосредственно рассчётом метрики

\subsection{Тестирование}

Для тестирования создаётся простой граф показанный на рисунке \ref{fig:sample_graph}
\begin{center}
    \centering
    \includegraphics[width=.6\linewidth]{extras/graph.png}
    \captionof{figure}{Граф для тестирования}
    \label{fig:sample_graph}
\end{center}

Модуль оптимизатора должен правильно посчитать метрики только для 2-х вершин: 2 и 3, так как они граничные. Их значения соответсвенно равны -2 и 1.

\begin{lstlisting}[caption=Вывод тестовой программы, style=text]
Metric for node 2 is -2
Metric for node 3 is 1
\end{lstlisting}

Результат верен, он показывает, что перемещение вершины 2 повысит связность, но по графу очевидно, что это внесёт только больший дисбаланс в количестве вершин между хранилищами, поэтому вершины и должны меняться местами.

\newpage
\nonumsection{Заключение}

В ходе выполнения курсового проекта была успешно исследована задача распределения вершин графа по гомогенным хранилищам и реализован ключевой компонент алгоритма Кернигана-Лина для оптимизации такого распределения.

На теоретическом уровне проведён анализ современных подходов к разбиению графов, включая как потоковые методы (Fennel, Streaming Graph Partitioning), так и методы оптимизации распределения (библиотека METIS). Установлено, что алгоритм Кернигана-Лина, несмотря на свою классическую природу, остаётся эффективным инструментом для уточнения разбиения графов, особенно в его оптимизированной граничной версии (BKL), которая значительно снижает вычислительные затраты.

На практическом уровне реализована вычислительная модель для расчёта метрики улучшения распределения $g_v$, являющейся основой алгоритма Кернигана-Лина. Разработаны:
\begin{itemize}
    \item Гибкая система представления графовых структур с поддержкой различных типов ключей вершин
    \item Классы для управления вершинами и их связями в рамках отдельных хранилищ
    \item Оптимизатор, вычисляющий метрику $g_v$ только для граничных вершин (реализация подхода Boundary KL)
    \item Демонстрационный пример, подтверждающий корректность работы реализованных компонентов
\end{itemize}

Ключевым достижением работы является реализация оптимизации Boundary KL, позволяющей работать только с вершинами, имеющими внешние связи, что значительно снижает вычислительную сложность алгоритма при сохранении качества оптимизации.

Полученные результаты подтверждают эффективность комбинированного подхода, при котором потоковые методы используются для начального распределения вершин, а алгоритм Кернигана-Лина — для последующей оптимизации. Такая стратегия позволяет достичь оптимального баланса между скоростью распределения и качеством разбиения графа, что особенно важно для распределённых графовых баз данных, работающих с большими объёмами данных в реальном времени.

Реализованная система может быть расширена для поддержки большего количества хранилищ, добавления полного цикла итераций алгоритма Кернигана-Лина и интеграции с реальными системами управления графовыми базами данных.
\newpage
\nonumsection{Приложение А}
Полные листинги кода программы.
\lstinputlisting[caption=graph.hpp, language=C++]{../include/graph.hpp}
\lstinputlisting[caption=interface\_bus.hpp, language=C++]{../include/interface\_bus.hpp}
\lstinputlisting[caption=bus.hpp, language=C++]{../include/bus.hpp}
\lstinputlisting[caption=storage.hpp, language=C++]{../include/storage.hpp}
\lstinputlisting[caption=optimizer.hpp, language=C++]{../include/optimizer.hpp}
\lstinputlisting[caption={main.cpp (Демонтстрация работы)}, language=C++]{../main.cpp}
%---------------------------------------------
\end{document}

