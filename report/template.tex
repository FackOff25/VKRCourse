\documentclass[a4paper, 12pt]{article}
\usepackage[T2A]{fontenc}
\usepackage[utf8]{inputenc}
\usepackage[english,russian]{babel}
\usepackage{titletoc}
\usepackage{graphicx}
\usepackage{array}
\usepackage{etoolbox}
\usepackage{subfig}
\usepackage{csvsimple}
\usepackage{float}
\usepackage{subcaption}
\usepackage{verbatim}
\usepackage{listings}
\usepackage{xcolor}
\usepackage[numbers]{natbib}
\usepackage{bibentry}
\usepackage{amsmath}
\usepackage{algorithm}
\usepackage{algpseudocode}
\usepackage[colorlinks,allcolors=black]{hyperref}
\definecolor{dkgreen}{rgb}{0,0.6,0}
\definecolor{gray}{rgb}{0.5,0.5,0.5}
\definecolor{mauve}{rgb}{0.58,0,0.82}
\definecolor{gray}{rgb}{0.4,0.4,0.4}
\definecolor{darkblue}{rgb}{0.0,0.0,0.6}
\definecolor{lightblue}{rgb}{0.0,0.0,0.9}
\definecolor{cyan}{rgb}{0.0,0.6,0.6}
\definecolor{darkred}{rgb}{0.6,0.0,0.0}

\include{includes/consts}
\usepackage[T2A]{fontenc}
\usepackage[utf8]{inputenc}
\usepackage[russian]{babel}

\usepackage[left=30mm, right=10mm, top=20mm, bottom=20mm]{geometry}

% Пакет для межстрочного интервала
\usepackage{setspace}
\onehalfspacing % Полуторный интервал

% Пакет для абзацного отступа
\usepackage{indentfirst}
\setlength{\parindent}{1.25cm} % Абзацный отступ 1.25 см

\addto{\captionsrussian}{\renewcommand{\contentsname}{Оглавление}}
\usepackage{listings}


\lstset{
	extendedchars=\true,
	inputencoding=utf8x,
	keepspaces=true,
	frame=tb,
	escapechar=`,
	xleftmargin=0.5cm,
	xrightmargin=0.5cm,
	columns=fixed,
	numbers=left,
	numbersep=4pt,
	showspaces=false,
	showstringspaces=false,
	breakatwhitespace=true,
	breaklines=true,
	basicstyle=\color{black}\small\sffamily,
	commentstyle=\color{gray}\itshape,
	stringstyle=\color{orange},
	numberstyle=\footnotesize\color{gray},
	keywordstyle=\color{blue}\bfseries,
	emphstyle={\color{blue}\bfseries},
	tabsize=2,
	texcl=true,
    mathescape=false
}

\lstset{
    language=bash,
    escapechar=`,
}

\lstdefinestyle{text}{
basicstyle=\small\ttfamily,
columns=flexible,
breaklines=true,
breakatwhitespace=true,
literate={\-}{{-\allowbreak}}1,
numbers=none,
postbreak=\mbox{\textcolor{red}{$\hookrightarrow$}\space},
mathescape=false
}

\lstdefinelanguage{cypher}
{
	morekeywords={
		MATCH, OPTIONAL, WHERE, NOT, AND, OR, XOR, RETURN, DISTINCT, ORDER, BY, ASC, ASCENDING, DESC, DESCENDING, UNWIND, AS, UNION, WITH, ALL, CREATE, DELETE, DETACH, REMOVE, SET, MERGE, SET, SKIP, LIMIT, IN, CASE, WHEN, THEN, ELSE, END,
		INDEX, DROP, UNIQUE, CONSTRAINT, EXPLAIN, PROFILE, START,
	}
}


\lstdefinelanguage{xml}
{
  morestring=[b]",
  morestring=[s]{>}{<},
  morecomment=[s]{<?}{?>},
  stringstyle=\color{black},
  identifierstyle=\color{darkblue},
  keywordstyle=\color{cyan},
  morekeywords={xmlns,version,type}% list your attributes here
}

\newcommand\YAMLcolonstyle{\color{red}\mdseries}
\newcommand\YAMLkeystyle{\color{black}\bfseries}
\newcommand\YAMLvaluestyle{\color{blue}\mdseries}

\makeatletter

% here is a macro expanding to the name of the language
% (handy if you decide to change it further down the road)
\newcommand\language@yaml{yaml}

\expandafter\expandafter\expandafter\lstdefinelanguage
\expandafter{\language@yaml}
{
  keywords={true,false,null,y,n},
  keywordstyle=\color{darkgray}\bfseries,
  basicstyle=\YAMLkeystyle,                                 % assuming a key comes first
  sensitive=false,
  comment=[l]{\#},
  morecomment=[s]{/*}{*/},
  moredelim=[l][\color{orange}]{\&},
  moredelim=[l][\color{magenta}]{*},
  moredelim=**[il][\YAMLcolonstyle{:}\YAMLvaluestyle]{:},   % switch to value style at :
  morestring=[b]',
  morestring=[b]",
  mathescape=false,
  literate =    {---}{{\ProcessThreeDashes}}3
                {>}{{\textcolor{red}\textgreater}}1     
                {|}{{\textcolor{red}\textbar}}1 
                {\ -\ }{{\mdseries\ -\ }}3,
}

% switch to key style at EOL
\lst@AddToHook{EveryLine}{\ifx\lst@language\language@yaml\YAMLkeystyle\fi}
\makeatother

\newcommand\ProcessThreeDashes{\llap{\color{cyan}\mdseries-{-}-}}

\lstdefinelanguage{docker}{
  keywords={FROM, RUN, COPY, ADD, ENTRYPOINT, CMD,  ENV, ARG, WORKDIR, EXPOSE, LABEL, USER, VOLUME, STOPSIGNAL, ONBUILD, MAINTAINER, HEALTHCHECK},
  identifierstyle=\color{black},
  sensitive=false,
  comment=[l]{\#},
  morestring=[b]',
  morestring=[b]"
}

\lstdefinelanguage{docker-compose}{
  keywords={image, environment, ports, container_name, ports, volumes, links},
  identifierstyle=\color{black},
  sensitive=false,
  comment=[l]{\#},
  morestring=[b]',
  morestring=[b]"
}
\lstdefinelanguage{docker-compose-2}{
  keywords={version, volumes, services},
  keywords=[2]{image, environment, ports, container_name, ports, links, build},
  keywordstyle=[2]\color{olive}\bfseries,
  identifierstyle=\color{black},
  sensitive=false,
  comment=[l]{\#},
  morestring=[b]',
  morestring=[b]"
}

\lstloadlanguages{SQL}

\newcolumntype{P}[1]{>{\centering\arraybackslash}p{#1}}
\def\textunderset#1#2{\leavevmode
  \vtop{\offinterlineskip\halign{%
    \hfil##\hfil\cr\strut#2\cr\noalign{\kern-.3ex}
    \hidewidth\strut#1\hidewidth\cr}}}
\def\signhrule{\raggedright\baselineskip30.0ex \vrule height 0.5pt width30mm depth0pt}

\dottedcontents{section}[2.5em]{\bfseries}{2em}{0.5pc}
\dottedcontents{subsection}[3em]{}{2em}{0.5pc}

\urlstyle{same}

%***natbib, bibentry***%
% Следующий код необходим для того, чтобы исправить конфликт между пакетами natbib+bibentry и стилем оформления ссылок согласно российскому ГОСТу cp1251gost705u
\ifx\undefined\selectlanguageifdefined
\def\selectlanguageifdefined#1{}\else\fi
\ifx\undefined\BibEmph
\def\BibEmph#1{\emph{#1}}\else\fi
\ifx\undefined\BibUrl
\def\BibUrl#1{\url{#1}}\else\fi
\ifx\undefined\BibAnnote
\def\BibAnnote#1{(#1)}\else\fi

\newcommand{\nonumsection}[1]{
\section*{#1}
\addcontentsline{toc}{section}{#1}
}
\begin{document}
\input{includes/titlepage}
\pagebreak
\tableofcontents
\newpage
% Основная часть 
\section{Описание лабораторной работы}
\subsection{цель}

Изучение процесса построения и обучения простой нейронной сети в PyTorch на примере задачи классификации ботов. Исследование влияния различных функций активации (Sigmoid, Tanh, ReLU) и оптимизаторов (SGD, SGD с моментом, RMSprop, Adam) на процесс обучения и качество модели. Определение оптимальной комбинации гиперпараметров для решения поставленной задачи классификации.
\subsection{Задание}

выполнить обучение и предсказание для задачи классификации/регрессии на примере:
\begin{enumerate}
    \item загрузка исходных данных (загрузить датасет, выполнить визуализацию исходных данных);
    \item подготовить данные для загрузки в модель (заполнить пропуски, выполнить замену текстовых 
    \item категориальных признаков, масштабировать значения при необходимости);
    \item построить модель MLP (from sklearn.neural\_network import MLPClassifier) и выполнить настройку параметров (количество слоев, число нейронов, функции активации);
    \begin{itemize}
        \item * дополнительные баллы при реализации MLP вручную без импорта готовой модели из библиотеки;
    \end{itemize}
    \item обучить модель и выполнить оценку ее адекватности при помощи метрик;
    \item ыполнить визуализацию результатов;
    \item составить отчет о проделанной работе в соответствии с требованиями кафедры.
\end{enumerate}
\newpage
% -------------------------------------------------------
\section{Выполнение лабораторной работы}
\subsection{Код}

Для выполнения лабораторной был написан следующий код.
\lstinputlisting[language=Python, caption=Код лабораторной работы]{../lab.py}

Вывод кода представлен ниже.
\lstinputlisting[style=text, caption=Вывод кода лабораторной работы]{../lab.txt}
\subsection{Визуализация данных}
\begin{center}
    \centering
    \includegraphics[width=1\linewidth]{../plots/1.png}
    \captionof{figure}{Гистограммы признаков}
\end{center}

\begin{center}
    \centering
    \includegraphics[width=.8\linewidth]{../plots/2.png}
    \captionof{figure}{Матрица корреляций}
\end{center}

\subsection{Визуализация результатов}
\begin{center}
    \centering
    \includegraphics[width=.8\linewidth]{../plots/3.png}
    \captionof{figure}{Матрица ошибок}
\end{center}
\begin{center}
    \centering
    \includegraphics[width=.8\linewidth]{../plots/4.png}
    \captionof{figure}{Кривая обучения}
\end{center}

\section{Контрольные вопросы}

\begin{enumerate}
    \item Что такое искусственный нейрон и как он работает?
    
    \textbf{Ответ:} Искусственный нейрон — математическая модель: $y = f(\sum (w_{i} \cdot x_{i}) + b)$. Принимает входные сигналы $x_i$, умножает на веса $w_i$, суммирует, добавляет смещение $b$, применяет нелинейную функцию активации $f$.

    \item Чем отличается линейная модель от многослойного персептрона?
    
    \textbf{Ответ:} Линейная модель — один слой, только линейные зависимости: $y = Wx + b$. MLP — несколько слоёв с нелинейными активациями: $y = f_n(\dots f_2(f_1(W_1x + b_1) + b_2)\dots)$, что позволяет моделировать сложные нелинейные зависимости.

    \item Для чего нужны функции активации?
    
    \textbf{Ответ:} 
        \begin{itemize}
            \item Вводят нелинейность (без них MLP эквивалентен линейной модели);
            \item Ограничивают выход;
            \item Управляют потоком градиентов;
            \item Cоздают разреженные представления (например, ReLU).
        \end{itemize}

    \item Как выбрать количество эпох и скрытых слоёв?
    
    \textbf{Ответ:} Эпохи: по кривой обучения или ранней остановке. Слои: начинать с 1–2, увеличивать при недообучении. Для простых задач — 0–1 слой, для сложных — 3+.

    \item Какие метрики применяются для оценки качества классификации?
    
    \textbf{Ответ:} Accuracy, Precision, Recall, F1-score, ROC-AUC. Для многоклассовой — macro/micro/weighted average. Выбор зависит от задачи (например, для медицины важен Recall).
\end{enumerate}

\newpage
\section{Вывод
}
Построен и обучен MLP для классификации. Данные успешно подготовлены. Модель с 2 скрытыми слоями (64→32 нейрона, ReLU) обучена Adam за 300 эпох с L2-регуляризацией.

Результаты: достигнута точность 0.99\%

Выводы: MLP эффективен для нелинейных задач. Ключевые факторы успеха — предобработка данных и правильный выбор гиперпараметров.

%---------------------------------------------
\end{document}

